\documentclass{article}
\usepackage[UTF8]{ctex}
\usepackage[backend=bibtex,style=numeric,natbib=true]{biblatex} % Use the bibtex backend with the authoryear citation style (which resembles APA)
\usepackage{graphicx}
\author{zhuangh7}
\title{How to Write an Acadmic Paper}
\addbibresource{one.bib} 

\begin{document}
\maketitle
\tableofcontents
\section{ABSTRACT}
There must be the moment when some one has to write a thesis.
But one may be confused, even though we have read more than one paper, 
we have no idea about how to write one.
Teach one how to write a good paper may be the duty of his mentor.
But "writing can be a trouble spot for both students and faculty"\cite{LearnIn2Hours}.

For students, many students dread writing, they may find it difficult to get their work published 
or write their dissertations.
And for teachers, it will always consume lots of time to help students write better.

This paper, try to discuss about oen topic \textbf{"writing a academic paper"}.



But a good paper is more than a record of words.
A good paper should be clearly and accessible.
People can eazily understand what's your idea, 
and feel pleasure to accept it.
People will enjoy a good paper and learn something from it! \cite{PeopleCanRead}


\section{Why We Have to Write Paper}
First of all, what is a academic paper? 
Why should we have the capability of writing academic paper?

\paragraph{Recording your knowledge}
Student should aware that papers are far more durable than programs \cite{AGreatResearchPaper}.
By writing a academic paper, one can nearly permanent records his work.
A academic paper can holds oen's knowledge and share it to the others.
When we reading papers, we learn from the antecessors, and construct our own idea.
So it's our duty to write down our knowledge for the human being.

\paragraph{Help to develop idea}
And writing paper can help one to develop the idea in the first place.
If one have the right procedure we talk below to do his work.
Writing paper will be the citical step for making both the idea and the question clearly.
So do not feel intimidated about writing papers.

\section{How Can We Write Good Paper}
\subsection{Procedure}
From Jones's\cite{AGreatResearchPaper} presentation, 
there are some useful procedures to help one write a good paper.
A big reason that one can't write a good paper is that he can not publish his idea clearly.
So, it is very import to identify your key idea.
In a newbie's mind, the way to do the work may like pic\ref{oldway}.

\begin{figure}[]
    \centering
    \includegraphics[width=1\textwidth]{1.png}    
    \caption{traditional procedure of work}
    \label{oldway}
    \includegraphics[width=1\textwidth]{2.png}    
    \caption{better procedure of work}
    \label{newway}
\end{figure}

But the procedure like pic\ref{newway} may be better.
Writing papers is a primary mechanism for doing research instead of just for reporting it \cite{AGreatResearchPaper}.
Follow the new way to do your research work, it can force you to be clear, 
and focu on your work.
And by writing paper before doing research, you can eazily to find what you don't understand.

\subsection{Principles}
There are some principles help one to write a good academic paper.
When one try to explain their works, or publish their ideas, 
they may follow the below principles:
\begin{itemize}
    \item Omit needless words.
    \item Pay attention to the rhythm of the paragraph.
    \item Group ideas into sentences in the most logical way.
\end{itemize}
Actually, you may be a little confused and the principle before may be difficlut to verify.
And there're many principles that people have and followed. 
Every professors may have their own principles to write a good
academic paper.
Fortunately, there still some principles more detaily.


\paragraph{Flow}
It should be clear how each sentence and paragraph relates to the adjacent ones. \cite{PeopleCanRead}
When a paper has a good flow style, readers can eazily follow your idea.
From what the problem is, to how to resolve the proble, and finally show the effiencient of the solution.
Follow this way, people can understand what's your idea and waht's your desion, and know your major contribution.

A good flow of your paragraphs is important for your paper to be eazily understood.
In more detail place,flow is still an important principle. 
The old to new principle can help you keep the follow inner one sentence.
In one sentence, show the information which more older before the lately one.This can help readers accept the information like time goes by and focus on the new things.

\paragraph{Coherence}
Good flow is not enough.
A good paper also needs good coherence.
For the paper, one paper should have only one key point,
and typically it should appears in the front of the paper.
For the paragraph,
the main idea of the paragraph should place in the first position.

Combining the flow principle and the coherence principle, the paper should not be too poorly to read.
\section{Conclusion}
If one want to contribute his knowledge to the world, he must have the capability to write a good academic.
A good academic paper can not only record one's knowledge, but also publish one's idea.
Therefore, we must learn to write an good paper,
and fortunately there are some simple principles to help us.
After hard effort, everyone can write a good paper.

\printbibliography
\end{document}
