\documentclass{article}
\usepackage[UTF8]{ctex}
\usepackage[backend=bibtex,style=numeric,natbib=true]{biblatex} % Use the bibtex backend with the authoryear citation style (which resembles APA)
\usepackage{graphicx}
\author{zhuangh7}
\title{How to Write an Acadmic Paper}
\addbibresource{one.bib} 

\begin{document}
\maketitle
\section{ABSTRACT}
There must be the moment when some one has to write a thesis.
But one may be confused, even though we have read more than one paper, 
we have no idea about how to write one.
I think teach you how to write a good paper is the duty of your mentor.
But "writing can be a trouble spot for both students and faculty"\cite{LearnIn2Hours}.

For students, many students dread writing, they may find it difficult to get their work published 
or write their dissertations.
And for teachers, it will always consume lots of time to help students write better.

Student should aware that papers are far more durable than programs \cite{AGreatResearchPaper}.
And writing paper can help one to develop the idea in the first place.
So do not feel intimidated about writing papers.
The reason one person needs to write a paper is always different.
Maybe their mentors ask them to, want they to publish their ideas to others.
A paper can hold one's ideas, and is more durable than a presentation.

But a good paper is more than a record of words.
A good paper should be clearly and accessible.
People can eazily understand what's your idea, 
and feel pleasure to accept it.
People will enjoy a good paper and learn something from it! \cite{d}


\section{main body}
\subsection{Procedure}
From Jones's\cite{AGreatResearchPaper} presentation, 
there are some useful procedures to help one write a good paper.
A big reason that one can't write a good paper is that he can not publish his idea clearly.
So, it is very import to identify your key idea.
In a newbie's mind, the way to do the work may like \ref{oldway}.

\begin{figure}[]
    \centering
    \includegraphics[width=1\textwidth]{1.png}    
    \caption{traditional procedure of work}
    \label{oldway}
    \includegraphics[width=1\textwidth]{2.png}    
    \caption{better procedure of work}
    \label{newway}
\end{figure}

But the procedure like \ref{newway} may be better.
Writing papers is a primary mechanism for doing research instead of just for reporting it \cite{AGreatResearchPaper}.
Follow the new way to do your research work, it can force you to be clear, 
and focu on your work.
And by writing paper before doing research, you can eazily to find what you don't understand.

\subsection{Principles}
There are some principles help one to write a good acadmic paper.
When one try to explain their works, or publish their ideas, 
they may follow the below principles:
\begin{itemize}
    \item Omit needless words.
    \item Pay attention to the rhythm of the paragraph.
    \item Group ideas into sentences in the most logical way.
\end{itemize}

\printbibliography
\end{document}
